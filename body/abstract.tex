\thispagestyle{empty} 
\section*{\zihao{3} \centering 海量数据的异常项检测及分类方法研究}
\vspace{0.5cm}

\noindent 
\textbf{摘\hspace{1em}要}:随着互联网的发展,人们正处于一个信息爆炸的时代。相比于过去的信息匮乏,面对现阶段海量的信息数据,人们淹没在数据中难以快速的制定合适的决策,使用简单的数据处理办法已经不能适用于现如今的需求。

 针对现状,本文采用不同数据处理方法来进行研究并对实现对给定的数据集进行数据检测分析,寻找能够在相同的硬件环境下提高海量数据处理效率的算法。提出了基于决策树的数据处理方法,即对数据进行特征提取、原始样本数据的预处理、构造决策树模型、进行数据挖掘预测等方面进行了深入讨论,并详细阐述了运用算子进行数据集处理的具体实现过程。利用rapidminer对数据处理过程进行可视化解析。通过对数据处理分析,所得数据具有较强的可读性、较好的可预测性。

设计并完成了数据在Web端的录入及储存在mysql数据库中,采用可视化工具rapidminer对mysql数据库中的数据进行读入,然后进行对读入数据进行预处理,即完成异常数据的检测及排除,在完成预处理后进一步进行所需属性的筛选,接着设置Label规则最后将处理后的数据传送给决策树模型,产生决策树可视化图像并最终根据决策树传出数据作出预测。

采用本文的数据处理方法可以大幅度提高数据使用效率,直观立体的反应海量数据对未来预测的指导性作用,明显优于传统数据处理方法。

\vspace{0.5cm}
\noindent
\textbf{关键词:}海量数据 ;数据处理;推荐算法;属性聚类;决策树
%\addcontentsline{toc}{section}{摘要}%摘要是否包括进目录中

\clearpage

\thispagestyle{empty} 
\section*{\songti\zihao{3} \centering \textbf{Research on the Detection and Classification of Exception in Massive Data
}}
   %用了Times New Roman字体来美化观感

\noindent 
\textbf{Abstract: }With the development of the Internet, people are in an era of information explosion. Compared to the lack of information in the past, in the face of massive information at this stage, it is difficult to develop appropriate decisions in the data, and the use of simple data processing can not be applied to today's needs.


In this paper, different data processing methods are used to study and realize the data detection and analysis of a given data set, and find an algorithm which can improve the efficiency of massive data processing in the same hardware environment. The data processing method based on decision tree is proposed, which is the feature extraction of the data, the preprocessing of the original sample data, the construction of the decision tree model, the data mining prediction and so on, and elaborates the use of the operator to carry out the data set Processing the specific implementation process. Using RapidMiner to visualize the data processing process. Through the data processing analysis, the obtained data has a strong readability, better predictability.

Design and complete the data in the Web side of the input and stored in the mysql database, the use of visual tools RapidMiner mysql database to read the data, and then read the data pre-processing, that is, the completion of abnormal data detection and exclusion, in the After the preprocessing, the filtering of the required attributes is carried out, and then the Label rule is set. Finally, the processed data is sent to the decision tree model, the decision tree visualization image is generated and finally the data is predicted based on the decision tree.

The data processing method of this paper can greatly improve the efficiency of data utilization, and the guidance function of intuitive stereo response to future forecasting is obviously superior to traditional data processing method.


\noindent 
\textbf{Keywords: }Massive Data; Data Processing; Recommendation Algorithm; Attribute Clustering;Decision tree
%\addcontentsline{toc}{section}{Abstract}%英文摘要是否包括进入目录
